\documentclass[11pt]{scrartcl}

%%%%%%%%%%%%%%%%%%%%%%%%%%%%%%%%%%%%%%%%%%%%%%%%%%%%%%%%%%%%%%%%%%%%%%%%%%%%
%NO INDENTATION, MORE ELEGANT
%%%%%%%%%%%%%%%%%%%%%%%%%%%%%%%%%%%%%%%%%%%%%%%%%%%%%%%%%%%%%%%%%%%%%%%%%%%%
\setlength{\parindent}{0pt}
%%%%%%%%%%%%%%%%%%%%%%%%%%%%%%%%%%%%%%%%%%%%%%%%%%%%%%%%%%%%%%%%%%%%%%%%%%%%

\usepackage[normalem]{ulem}

\usepackage{amsmath,amssymb,latexsym,soul,amsthm}
\usepackage{cite,amsfonts}
\usepackage{color,enumitem,graphicx}
\usepackage[colorlinks=true,urlcolor=blue,
=red,linkcolor=blue,linktocpage,pdfpagelabels,
bookmarksnumbered,bookmarksopen]{hyperref}
\usepackage[english]{babel}

%%%%%%%%%%%%%%%%%%%%%%%%%%%%%%%%%%%%%%%%%%%%%%%%%%%%%%%%%%%%%%%%%%%%%%%%%%%% BETTER TITLE %%%%%%%%%%%%%%%%%%%%%%%%%%%%%%%%%%%%%%%%%%%%%%%%%%%%%%%%%%%%%%%%%%%%%%%%%%%
\usepackage{authblk}


%%%%%%%%%%%%%%%%%%%%%%%%%%%%%%%%%%%%%%%%%%%%%%%%%%%%%%%%%%%%%%%%%%%%%%%%%%%%
\let\equation\gather                          %% See tabu and hyperref docs
\let\endequation\endgather
%%%%%%%%%%%%%%%%%%%%%%%%%%%%%%%%%%%%%%%%%%%%%%%%%%%%%%%%%%%%%%%%%%%%%%%%%%%%

%%%%%%%%%%%%%%%%%%%%%%%%%%%%%%%%%%%%%%%%%%%%%%%%%%%%%%%%%%%%%%%%%%%%%%%%%%%%
%LARGER AREA THAN DEFAULT
%%%%%%%%%%%%%%%%%%%%%%%%%%%%%%%%%%%%%%%%%%%%%%%%%%%%%%%%%%%%%%%%%%%%%%%%%%%%
\usepackage[a4paper,left=1.0in,right=1.0in,top=3cm,bottom=3cm]{geometry}
%%%%%%%%%%%%%%%%%%%%%%%%%%%%%%%%%%%%%%%%%%%%%%%%%%%%%%%%%%%%%%%%%%%%%%%%%%%%

%%%%%%%%%%%%%%%%%%%%%%%%%%%%%%%%%%%%%%%%%%%%%%%%%%%%%%%%%%%%%%%%%%%%%%%%%%%%
%FOR PREPRINTS, LINES ARE MORE SPACED THAN DEFAULT
%%%%%%%%%%%%%%%%%%%%%%%%%%%%%%%%%%%%%%%%%%%%%%%%%%%%%%%%%%%%%%%%%%%%%%%%%%%%
\renewcommand{\baselinestretch}{1.1}
%%%%%%%%%%%%%%%%%%%%%%%%%%%%%%%%%%%%%%%%%%%%%%%%%%%%%%%%%%%%%%%%%%%%%%%%%%%%



%%%%%%%%%%%%%%%%%%%%%%%%%%%%%%%%%%%%%%%%%%%%%%%%%%%%%%%%%%%%%%%%%%%%%%%%%%%
%REFERENCE CHECK AND LABELS
%%%%%%%%%%%%%%%%%%%%%%%%%%%%%%%%%%%%%%%%%%%%%%%%%%%%%%%%%%%%%%%%%%%%%%%%%%%
%\usepackage[hyperpageref]{backref}
%\usepackage{refcheck}
%\usepackage{showkeys}
%%%%%%%%%%%%%%%%%%%%%%%%%%%%%%%%%%%%%%%%%%%%%%%%%%%%%%%%%%%%%%%%%%%%%%%%%%%%

%%%%%%%%%%%%%%%%%%%%%%%%%%%%%%%%%%%%%%%%%%%%%%%%%%%%%%%%%%%%%%%%%%%%%%%%%%%%
%FONTS
%%%%%%%%%%%%%%%%%%%%%%%%%%%%%%%%%%%%%%%%%%%%%%%%%%%%%%%%%%%%%%%%%%%%%%%%%%%%
%\usepackage{mathrsfs}
\usepackage[T1]{fontenc}
\usepackage{lmodern}
\usepackage{microtype}
%%%%%%%%%%%%%%%%%%%%%%%%%%%%%%%%%%%%%%%%%%%%%%%%%%%%%%%%%%%%%%%%%%%%%%%%%%%%

%%%%%%%%%%%%%%%%%%%%%%%%%%%%%%%%%%%%%%%%%%%%%%%%%%%%%%%%%%%%%%%%%%%%%%%%%%%%
%GOOD PRACTICE FOR MATHEMATICAL FORMULAS
%%%%%%%%%%%%%%%%%%%%%%%%%%%%%%%%%%%%%%%%%%%%%%%%%%%%%%%%%%%%%%%%%%%%%%%%%%%%
\usepackage{fixmath}
%%%%%%%%%%%%%%%%%%%%%%%%%%%%%%%%%%%%%%%%%%%%%%%%%%%%%%%%%%%%%%%%%%%%%%%%%%%%

%%%%%%%%%%%%%%%%%%%%%%%%%%%%%%%%%%%%%%%%%%%%%%%%%%%%%%%%%%%%%%%%%%%%%%%%%%%%
%TIKZ COMMUTATIVE DIAGRAMS
%%%%%%%%%%%%%%%%%%%%%%%%%%%%%%%%%%%%%%%%%%%%%%%%%%%%%%%%%%%%%%%%%%%%%%%%%%%%
%\usepackage{tikz-cd}
%%%%%%%%%%%%%%%%%%%%%%%%%%%%%%%%%%%%%%%%%%%%%%%%%%%%%%%%%%%%%%%%%%%%%%%%%%%%

%%%%%%%%%%%%%%%%%%%%%%%%%%%%%%%%%%%%%%%%%%%%%%%%%%%%%%%%%%%%%%%%%%%%%%%%%%%%
%COMMENTS INSIDE COLORED BOXES
%%%%%%%%%%%%%%%%%%%%%%%%%%%%%%%%%%%%%%%%%%%%%%%%%%%%%%%%%%%%%%%%%%%%%%%%%%%%
%\usepackage[most]{tcolorbox}
%
%\tcbset{
%	frame code={}
%	center title,
%	left=0pt,
%	right=0pt,
%	top=0pt,
%	bottom=0pt,
%	colback=gray!70,
%	colframe=white,
%	width=\dimexpr\textwidth\relax,
%	enlarge left by=0mm,
%	boxsep=5pt,
%	arc=0pt,outer arc=0pt,
%}




%%%%%%%%%%%%%%%%%%%%%%%%%%%%%%%%%%%%%%%%%%%%%%%%%%%%%%%%%%%%%%%%%%%%%%%%%%%%%%%
%THEOREMS ETC
%%%%%%%%%%%%%%%%%%%%%%%%%%%%%%%%%%%%%%%%%%%%%%%%%%%%%%%%%%%%%%%%%%%%%%%%%%%%%%%

\numberwithin{equation}{section}

\newtheorem{theorem}{Theorem}[section]
\newtheorem{lemma}[theorem]{Lemma}
\newtheorem{proposition}[theorem]{Proposition}
\newtheorem{corollary}[theorem]{Corollary}
\theoremstyle{definition}
\newtheorem{definition}[theorem]{Definition}
\theoremstyle{remark}
\newtheorem{remark}[theorem]{Remark}
\newtheorem{ex}[theorem]{Example}

\newtheorem*{ack}{Acknowledgments}
%%%%%%%%%%%%%%%%%%%%%%%%%%%%%%%%%%%%%%%%%%%%%%%%%%%%%%%%%%%%%%%%%%%%%%%%%%%%%%

\renewcommand{\ge}{\varepsilon}
%\DeclareMathOperator{\esssup}{ess\,sup}

\newcommand{\R}{\mathbb{R}}
\newcommand{\C}{\mathbb{C}}

%\newcommand{\h}{H^1(\mathbb{R}_{+}^{N+1})}

\newcommand{\h}{H^1_V(\mathcal{M})}
%\usepackage{amssymb,amsmath,amsthm}

%\usepackage{showframe}

%\usepackage[utf8]{inputenc}
%\usepackage{lmodern}
%\usepackage{indentfirst}

\usepackage{todonotes}

%\usepackage[left=2.3cm, right=2.3cm, top=3.0cm, bottom=3.5cm]{geometry}
%\usepackage{fullpage}





%\newcommand{\eps}{\varepsilon}

%\newcommand{\R}{\mathbb{R}}
%\newcommand{\Q}{\mathbb{Q}}
%\newcommand{\Z}{\mathbb{Z}}
%\newcommand{\N}{\mathbb{N}}

%\newcommand{\F}{{\cal F}}

\newcommand{\cA}{{\mathcal A}}
\newcommand{\cB}{{\mathcal B}}
\newcommand{\cC}{{\mathcal C}}
\newcommand{\cD}{{\mathcal D}}
\newcommand{\cE}{{\mathcal E}}
\newcommand{\cF}{{\mathcal F}}
\newcommand{\cG}{{\mathcal G}}
\newcommand{\cH}{{\mathcal H}}
\newcommand{\cI}{{\mathcal I}}
\newcommand{\cJ}{{\mathcal J}}
\newcommand{\cK}{{\mathcal K}}
\newcommand{\cL}{{\mathcal L}}
\newcommand{\cM}{{\mathcal M}}
\newcommand{\cN}{{\mathcal N}}
\newcommand{\cO}{{\mathcal O}}
\newcommand{\cP}{{\mathcal P}}
\newcommand{\cQ}{{\mathcal Q}}
\newcommand{\cR}{{\mathcal R}}
\newcommand{\cS}{{\mathcal S}}
\newcommand{\cT}{{\mathcal T}}
\newcommand{\cU}{{\mathcal U}}
\newcommand{\cV}{{\mathcal V}}
\newcommand{\cW}{{\mathcal W}}
\newcommand{\cX}{{\mathcal X}}
\newcommand{\cY}{{\mathcal Y}}
\newcommand{\cZ}{{\mathcal Z}}
\newcommand{\al}{\alpha}
\newcommand{\be}{\beta}
\newcommand{\ga}{\gamma}
\newcommand{\de}{\delta}
\newcommand{\De}{\Delta}
\newcommand{\Ga}{\Gamma}
\newcommand{\Om}{\Omega}
\newcommand{\M}{\mathcal{M}}
\newcommand{\RR}{\mathbb R}
\newcommand{\NN}{\mathbb N}
\newcommand{\N}{\mathbb N}
\newcommand{\Z}{\mathbb Z}
\newcommand{\Q}{\mathbb Q}
%\newcommand{\R}{\mathbb R}
%\newcommand{\C}{\mathbb C}
\DeclareMathOperator{\essinf}{ess\, inf}
\DeclareMathOperator{\esssup}{ess\, sup}
\DeclareMathOperator{\meas}{meas}

\newcommand{\weakto}{\rightharpoonup}
\newcommand{\pa}{\partial}

\newcommand{\tX}{\widetilde{X}}
\newcommand{\tu}{\widetilde{u}}
\newcommand{\tv}{\widetilde{v}}
\newcommand{\tcV}{\widetilde{\cV}}

\newcommand{\wh}{\widehat}
\newcommand{\wti}{\widetilde}
\newcommand{\cTto}{\stackrel{\cT}{\longrightarrow}}
\newcommand{\ctto}{\stackrel{\tau}{\longrightarrow}}

\numberwithin{equation}{section}


\DeclareMathOperator*{\supp}{supp}

\DeclareSymbolFont{rsfs}{U}{rsfs}{m}{n}
\DeclareSymbolFontAlphabet{\mathscr}{rsfs}


\begin{document}

\title{A sample paper}
\author{Simone Secchi\thanks{Email address: \texttt{Simone.Secchi@unimib.it}}}
\affil{\small Dipartimento di Matematica e Applicazioni Universit\`a degli Studi di Milano-Bicocca, via Roberto Cozzi 55, I-20125, Milano, Italy}

\maketitle

\begin{abstract}
In this paper we 
\end{abstract}

\section{Introduction}

We consider the equation
\begin{equation}
(I-\Delta)^s u = a(x) |u|^{p-2}u \quad \hbox{in \(\mathbb{R}^N\)},
\end{equation}
where \(a \in L^\infty(\mathbb{R}^N)\) and $2<p<2^\star = 2N/(N-2)\) if \(N \geq 3\), and \(p>2\) if \(N=1\) or \(N=2\).

\section{The variational setting}

\begin{lemma}
Let \(u \in L^\infty(\mathbb{R}^N)\). There results
\begin{equation} \label{eq:2.1}
\operatorname{ess\, sup} u = \sup \left\{ \int_{\mathbb{R}^N} u \varphi \mid \varphi \in L^1(\mathbb{R}^N), \ \varphi \geq 0, \ \int_{\mathbb{R}^N} \varphi =1 \right\}.
\end{equation}
\end{lemma}
\begin{proof}
Whenever \(\varphi \in L^1(\mathbb{R}^N)\), \(\varphi \geq 0\), \( \int_{\mathbb{R}^N} \varphi =1\), we compute
\begin{equation*}
\int_{\mathbb{R}^N} u \varphi \leq \operatorname{ess\, sup} u \int_{\mathbb{R}^N} \varphi = \operatorname{ess\, sup} u.
\end{equation*}
Hence
\begin{equation} \label{eq:2.2}
\operatorname{ess\, sup} u \geq \sup \left\{ \int_{\mathbb{R}^N} u \varphi \mid \varphi \in L^1(\mathbb{R}^N), \ \varphi \geq 0, \ \int_{\mathbb{R}^N} \varphi =1 \right\}.
\end{equation}
On the other hand, if we set
\begin{equation*}
\sup \left\{ \int_{\mathbb{R}^N} u \varphi \mid \varphi \in L^1(\mathbb{R}^N), \ \varphi \geq 0, \ \int_{\mathbb{R}^N} \varphi =1 \right\} = b
\end{equation*}
and we assume that \(\operatorname{ess\, sup} u >  b\),
then for some \(\delta>0\) we can say  that the set \(\Omega = \left\{ x \in \mathbb{R}^N \mid u(x) \geq b + \delta \right\}\) has positive measure. Let us define \(\varphi = \chi_\Omega / \mathcal{L}^N(\Omega)\), so that 
\begin{equation*}
\int_{\mathbb{R}^N} u \varphi = \frac{1}{\mathcal{L}^N(\Omega)} \int_\Omega u \geq b+\delta,
\end{equation*}
contrary to \eqref{eq:2.2}. This completes the proof.
\end{proof}

\begin{thebibliography}{99}


\bibitem{VPricceri} \textsc{B. Ricceri},\emph{ A general variational principle and some of its applications},
 J. Comput. Appl. Math., \textbf{ 113}  (2000), 401--410.

\bibitem{struwe} \textsc{M. Struwe}, Variational methods,
Applications to nonlinear partial differential equations and Hamiltonian systems,
\emph{ Ergebnisse der Mathematik und ihrer Grenzgebiete},  3,
\emph{ Springer Verlag}, Berlin--Heidelberg (1990).

\bibitem{Wi} \textsc{M. Willem}, {Minimax Theorems}, Birkh\"auser, Basel (1999).

\bibitem{Gigio1} \textsc{L. S. Yu}, \emph{Nonlinear $p$--Laplacian problems on unbounded domains}, Proc. Amer. Math. Soc.
\textbf{115} (1992), 1037--1045.

\end{thebibliography}

\end{document}
